\newpage
\section{Merkblatt}
\subsection{Überblick: Spoken Language Systems}
Spoken Language Processing ist ein Schnittbereich aus Psychologie, Informatik, Linguistik und Elektrotechnik. \\
Ein Spoken Language System braucht \textbf{Spracherkennung}, \textbf{Sprachsynthese} und eine \textbf{Understanding-Komponente} (z.Bsp. Dialog oder Übersetzung)! \\
Wichtig ist dabei auch \textbf{Domainenwissen}, damit das System spezifische Unterscheidungen beurteilen und angemessen reagieren kann. \\
Herausforderungen bei der Entwicklung dieser Teilsysteme:
\begin{itemize}
\item Robustfähigkeit
\item Integrierbarkeit
\item Flexibilität
\item Effizienz
\end{itemize}

\subsection{Erster Schritt: Wie kommt die Sprache in den Computer?}
Mithilfe eines (Kondensator)Mikrofons wird die natürliche Sprache aufgenommen und dann vorverarbeitet, damit der Computer das analoge Signal versteht und bearbeiten kann. Das bedeutet:
\begin{itemize}
\item Signaldigitalisierung durch \textbf{Quantisierung} ($2^{16}$ Intervalle) und \textbf{Sampling} \\ (Schutz vor Aliasing: Filter oder Nyquist Theorem)
\item Digitale Signal Vorverarbeitung mithilfe von Merkmalsextraktion \\(Entnehme aus dem Signal Merkmale die für die Spracherkennung hilfreich sind)
\end{itemize}
\subsubsection{Repräsentationsarten}
Definition: Digitale Repräsentation von Sprache bedeutet, ein akustisches Signal als eine Sequenz von Nummern zu representieren.
\begin{itemize}
\item Direkte Representation: Repräsentiere die Sprachwelle so akkurat wie möglich um sie später wieder in ein akustisches Signal umzuwandeln
\item Parametrische Repräsentation: Wähle eine Menge von Parametern innerhalb eines Modells zur Darstellung des Signals. (Stark abhängig vom Verwendungszweck: Sprachsynthese, Sprachübersetzung, Sprachkodierung)
\end{itemize}

\subsubsection{Merkmalsextraktion}
Warum? \begin{itemize}
 \item Wir wollen wichtige phonetische Information hervorheben
 \item Macht es einfacher für den Computer mit dem Signal zu arbeiten, weniger Speicher
 \item Optimiert Generalisierfähigkeit
 \end{itemize}
Wie? \begin{itemize}
\item Es ist schwierig nützliche Information in der time domain herauszuziehen
\item Wir nehmen uns das menschliche Ohr als Vorbild: Frequenzanalyse
\item Frequenzanalyse beschleunigt die Signalverarbeitung und erleichtert das Verständnis
\end{itemize}

